\begin{itemize}
\item Each process has a unique process identifier (\texttt{pid})
\item Shells also have job control
\end{itemize}
\subsubsection{Foreground and Background}
\begin{itemize}
\item \emph{foreground} processes will hold the terminal
\item Only one can run in a terminal at a time. e.g \texttt{/usr/bin/xeyes}
\item Suffixing \texttt{\&} makes it run in the \emph{background}. e.g  \texttt{/usr/bin/xeyes \&}
\end{itemize}
\subsubsection{Checking processes}
\begin{itemize}
\item \texttt{ps} lists processes
\item \texttt{ps -u root} lists root processes
\item \texttt{ps -opid,cmd} shows only pid and commands
\end{itemize}
\subsubsection{Signals}
\begin{itemize}
\item Signals allow ways to send messages to running processes
\item \texttt{man 7 signal} gives you list of signals
\item \texttt{kill} sends signals to a process
\item \texttt{kill \textit{pid}} sends \texttt{SIGINT} to \textit{pid}
\item \texttt{SIGINT} is created when you hit \keys{\ctrl} \keys{c}
\item Signals can be caught (run a custom handler), ignored (don't
  respond to it)
\end{itemize}
\subsubsection{Jobs}
\begin{itemize}
  \item \emph{Jobs} as pipelines started from a shell.
  \item The \texttt{jobs} command gives a list of jobs in the shell.
  \item Job identifiers are prefixed with a \%.
  \item \texttt{kill -9 \%1} will kill job 1.
  \item \keys{\ctrl} \keys{z} will stop a foreground job (sends \texttt{SIGSTOP}).
  \item Can be backgrounded using \texttt{bg}
  \item Can be foregrounded using \texttt{fg}
  \item Can be disowned using \texttt{fg} (no longer a shell job)
\end{itemize}
