\documentclass{../refsheet}

\usepackage{listings}
\usepackage{graphicx}
\usepackage{textcomp}
\usepackage{menukeys}
\usepackage{epigraph}

\renewmenumacro{\menu}[>]{shadowedroundedkeys}
%% \renewcommand{\familydefault}{\sfdefault}
%% \renewcommand*\rmdefault{iwona}

\title{Basic UNIX Refcard}
\author{The Lyc\ae{}um, 2018}
\date{}


\definecolor{lightgray}{rgb}{0.9,0.9,0.9}
\definecolor{darkgray}{rgb}{0.2,0.2,0.1}

\setlength\epigraphwidth{.3\textwidth}
\setlength\epigraphrule{0pt}
\renewcommand{\epigraphflush}{flushleft}
\renewcommand{\sourceflush}{flushright}

\def\quoteattr#1#2{\setbox0=\hbox{#2}#1\tabto{\dimexpr\linewidth-\wd0}\box0}

\lstset{ %
  %% backgroundcolor=\color{lightgray},   % choose the background color; you
  %%                               % must add \usepackage{color} or
  %%                               % \usepackage{xcolor}
  basicstyle=\footnotesize \ttfamily,
  escapeinside={\%*}{*)},
  commentstyle=\color{magenta} \textit,
  keywordstyle=\color{blue} \textbf,
  rulecolor=\color{gray},
  language=python,
  showspaces=false,
  showstringspaces=false,
  frame=single
}

\begin{document}
\maketitle
\epigraph{\textit{This is the Unix philosophy: Write programs that do one
  thing and do it well. Write programs to work together. Write
  programs to handle text streams, because that is a universal
  interface.}}{\textit{-- Doug McIlroy}}
\section{Navigation}
Most shells come with command line editing capability. Useful for
quick editing
\begin{center}
\begin{tabular}{ r  l }
  Operation & Key combination \\
  \hline   
    forward-word & \keys{Meta} \keys{f} \\
    backward-word & \keys{Meta} \keys{b} \\
    beginning-of-line & \keys{\ctrl} \keys{a} \\
    end-of-line & \keys{\ctrl} \keys{e} \\
    delete-till-end-of-line & \keys{\ctrl} \keys{k} \\
    delete-line & \keys{\ctrl} \keys{a} \keys{\ctrl} \keys{k}\\
    delete-forward-word & \keys{Meta} \keys{d} \\
    delete-backward-word & \keys{Meta} \keys{backspace} \\
    transpose-words & \keys{Meta} \keys{t} \\
    \hline 
    Paste (yank) & \keys{\ctrl} \keys{y} \\
    Start marking & \keys{\ctrl} \keys{\SPACE}  \\
    Copy marked region & \keys{Meta} \keys{w} \\
    \hline 
    history-previous & \keys{\ctrl} \keys{p} \\
    history-next & \keys{\ctrl} \keys{n}  \\
    history-search-backward & \keys{\ctrl} \keys{r} \\
    history-search-forward & \keys{\ctrl} \keys{s}  \\
    beginning-of-history & \keys{Meta} \keys{\textless} \\
    end-of-history & \keys{Meta} \keys{\textgreater} \\    
    \hline 
    Undo & \keys{\ctrl} \keys{\textunderscore} \\
    Undo-line & \keys{\ctrl} \keys{r} \\
\end{tabular}
\end{center}

\noindent\rule{\linewidth}{0.05ex}

\section{Useful shell expressions}
\begin{center}
  \begin{tabular} { l | l }
    Expression & Use \\
    \hline
    \textasciicircum{}x\textasciicircum{}y & Run last command with \texttt{x} changed to \texttt{y} \\
    \$\_ & Last argument of previous command \\
    !! & Last command \\
  \end{tabular}

\end{center}

The \keys{Tab} key completes commands, files and can also be programmed.\\

\noindent\rule{\linewidth}{0.05ex}

\section{Terminology}
\begin{itemize}
\item A \textbf{shell} is an interactive program which will read input
  from a user and run requested commands
\item A \textbf{command} is an executable program you can run. e.g. \textit{/bin/ls}
\item \textbf{Options} is an optional switch to a
  command. e.g. \textit{/bin/ls \textbf{-s}}. Also long options \textit{/bin/ls \textbf{-{}-size}}
\item Multiple short options can be combined. e.g. \textit{/bin/ls -ls}
\item \textbf{Arguments} are passed to commands to tell it what to
  work on. e.g. \textit{/bin/ls -l \textbf{poem1.txt}}
\item A \textbf{process} is a program in execution.
\item A \textbf{job} is a pipeline of commands executed from a shell
\item A shell variable is variable maintained by the shell
\end{itemize}




\noindent\rule{\linewidth}{0.05ex}

\section{Some commands}
Use \texttt{man \textit{command}} for information on usage.
\begin{center}

  \begin{tabular}{l l l}
    Command & function \\
    \hline
    cat & Displays contents for files\\
    cd & Change directory \\
    cut & Cut columns \\
    date & Prints current or another date\\
    du & Shows disk usage\\
    echo & echoes its arguments \\
    file & Guess what a file is\\
    grep & search for patterns \\
    head & Show top lines \\
    ls & List files\\
    man & show help for command\\
    mkdir & Make directory \\
    mount & Prints mounted filesystems\\
    od & octal dump \\
    paste & Merges columns \\
    ps & Lists processes\\
    pwd & show working directory \\
    rev & Reverse lines \\
    sed & Stream editor \\
    seq & Prints sequence of numbers \\
    sort & Sort lines \\
    tail & Show bottom lines \\
    tr & Translate character sets \\
    tty & Prints terminal file\\
    uniq & unique lines of a sorted file \\
    wc & Shows word count\\
    who & Prints logged in users\\
    mv & Rename files \\
    cp & Copy files \\
    rm & Delete files \\
    find & Find files \\
    \end {tabular}
\end{center}

\noindent\rule{\linewidth}{0.05ex}

\section{Globbing}
\begin{enumerate}
\item \texttt{*} Any string of characters \\
\item \texttt{?} Any single character \\
\item \texttt{\{} and \texttt{\}} specifies a list of alternatives
\end{enumerate}

\noindent\rule{\linewidth}{0.05ex}

\section{Quoting}
\begin{itemize}
\item \texttt{``} and \texttt{'} will suppress special meanings of
  symbols. e.g. \texttt{ls ``*.txt''}  vs. \texttt{ls *.txt}.
\item \texttt{\textbackslash} will quote a single character. So the above can be
  \texttt{ls \textbackslash*.txt}.
\end{itemize}
\noindent\rule{\linewidth}{0.05ex}

\section{Redirection}
\begin{itemize}
\item \texttt{\textgreater} redirects output of a command. Overwrites
  target.
\item \texttt{\textgreater\textgreater} redirects output of a
  command. Appends to target.
\item \texttt{\textless} redirects input into a program
\item \texttt{\textless\textless} is used to create a \emph{here document}.
\end{itemize}
\subsection{Pipes}
\begin{itemize}
\item \texttt{|} connects the output of one program to the input of
  the next. e.g. \texttt{ls | sort}
\item e.g. \texttt{ls -1 | wc -l} counts the number of files in the
  current directory.
\end{itemize}

\noindent\rule{\linewidth}{0.05ex}

\section{File descriptors}
\begin{itemize}
\item stdin (\texttt{0}), stdout (\texttt{1}) and stderr(\texttt{2})
\item \texttt{cmd \textgreater{} output 2\textgreater{} error}
\item \texttt{cmd \textgreater{} output 2\textgreater{}\&1} puts both
  into \texttt{output}
\end{itemize}

\noindent\rule{\linewidth}{0.05ex}

\section{Substitution}
\subsection{command substitution}
\begin{itemize}
\item \texttt{\$(} and \texttt{)} will run the enclosed command and
  replace output with that.
\item \texttt{rm \$(find . -size +10M)}. Find all files larger than
  10MB and delete them. 
\end{itemize}

\noindent\rule{\linewidth}{0.05ex}

\section{Multiprocessing}
\begin{itemize}
\item Each process has a unique process identifier (\texttt{pid})
\item Shells also have job control
\end{itemize}
\subsubsection{Foreground and Background}
\begin{itemize}
\item \emph{foreground} processes will hold the terminal
\item Only one can run in a terminal at a time. e.g \texttt{/usr/bin/xeyes}
\item Suffixing \texttt{\&} makes it run in the \emph{background}. e.g  \texttt{/usr/bin/xeyes \&}
\end{itemize}
\subsubsection{Checking processes}
\begin{itemize}
\item \texttt{ps} lists processes
\item \texttt{ps -u root} lists root processes
\item \texttt{ps -opid,cmd} shows only pid and commands
\end{itemize}
\subsubsection{Signals}
\begin{itemize}
\item Signals allow ways to send messages to running processes
\item \texttt{man 7 signal} gives you list of signals
\item \texttt{kill} sends signals to a process
\item \texttt{kill \textit{pid}} sends \texttt{SIGINT} to \textit{pid}
\item \texttt{SIGINT} is created when you hit \keys{\ctrl} \keys{c}
\item Signals can be caught (run a custom handler), ignored (don't
  respond to it)
\end{itemize}
\subsubsection{Jobs}
\begin{itemize}
  \item \emph{Jobs} as pipelines started from a shell.
  \item The \texttt{jobs} command gives a list of jobs in the shell.
  \item Job identifiers are prefixed with a \%.
  \item \texttt{kill -9 \%1} will kill job 1.
  \item \keys{\ctrl} \keys{z} will stop a foreground job (sends \texttt{SIGSTOP}).
  \item Can be backgrounded using \texttt{bg}
  \item Can be foregrounded using \texttt{fg}
  \item Can be disowned using \texttt{fg} (no longer a shell job)
\end{itemize}

\noindent\rule{\linewidth}{0.05ex}

\section{Shell variables}
\begin{itemize}
\item Shell variables are variables maintained by the shell
\item \texttt{FOO=1} sets the value of variable \texttt{FOO} to \texttt{1}
\item No spaces around the \texttt{=} and if the value has a space, it
  should be quoted. (e.g. \texttt{FOO=''Brian Kernighan''}).
\item Can be accessed later using the \texttt{\$} sigil. e.g. \texttt{\$\{FOO\}}
\item Variables are expanded inside double quotes (\texttt{``}). Not
  inside \texttt{'}
\item Subshells will not have copies of the variable unless exported
  (e.g. \texttt{export FOO})
\item Some special variables
  \begin{center}
    \begin{tabular}{ r  l }
      PS1 & Shell prompt \\
      PATH & Paths to search for commands \\
      SHLVL & Current shell level \\
      MANPATH & Paths to search for manual pages \\
      \texttt{?} & Exit status of last command \\
      \texttt{\_} & Last argument to previous command \\
      \texttt{!} & Process id of last background command \\
    \end {tabular}
  \end{center}
\end{itemize}

\noindent\rule{\linewidth}{0.05ex}




\begin{center}
\includegraphics[scale=0.4]{../images/parthenon-callig.png}
\end{center}
For more information on courses which the Lyc\ae{}um conducts, please
visit \texttt{http://thelycaeum.in}. \\To stay notified on events and
courses, sign up for our mailer at \texttt{http://thelycaeum.in/resources.html}. \\For further
information, please email \\\texttt{noufal@nibrahim.net.in} or call
\textit{+91-9567091835}.
\vspace{0.5cm}

The Lyc\ae{}um conducts a 4 month long mentoring course for fresher
programmers and offers placement assistance. Please contact if you're
interested. The next batch starts in \rule{4cm}{0.1ex}.

    \textcolor{lightgray}{\noindent\rule{\linewidth}{0.05ex}}
\footnotesize All content \textcopyright The Lyc\ae{}um 2016,17,18. Version \input{version.tex}

\end{document}

