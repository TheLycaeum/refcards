\documentclass{refsheet}

\usepackage{listings}
\usepackage{graphicx}
\usepackage{textcomp}

%% \renewcommand{\familydefault}{\sfdefault}
%% \renewcommand*\rmdefault{iwona}

\title{Python 3 Refcard}
\author{The Lyc\ae{}um, 2016}
\date{}


\definecolor{lightgray}{rgb}{0.9,0.9,0.9}
\definecolor{darkgray}{rgb}{0.2,0.2,0.1}

\lstset{ %
  %% backgroundcolor=\color{lightgray},   % choose the background color; you
  %%                               % must add \usepackage{color} or
  %%                               % \usepackage{xcolor}
  basicstyle=\footnotesize \ttfamily,
  commentstyle=\color{gray} \textit,
  keywordstyle=\color{blue} \textbf,
  rulecolor=\color{gray},
  language=python,
  showspaces=false,
  showstringspaces=false,
  frame=single
}

\begin{document}
\maketitle
\section{Types}
\subsection{Numeric types}
Integers, Floats, complex numbers \\
e.g. \texttt{10, 54, 2.5, 3.0, 3+4j, 6-2j} \\
\textit{Operations:} \texttt{+, -, *, /, \%, //, **, abs()}
\begin{lstlisting}
# x and y are complex numbers
x = 3+4j
print (x + 2)  # 5+4j
print (x.real, x.imag) # 3.0, 4.0
print (x.conjugate()) # 3-4j
print (abs(x)) # 5
print (5/2, 5//2) # 2.5 2
print (5**2) # 25
print (abs(-5)) # 5
\end{lstlisting}

\subsection{Strings}
\texttt{``Example''}, \texttt{'example'}, \texttt{""" \\
Multiline \\
string """}\\
\textit{Operations:} \texttt{+, *, len()}
\begin{lstlisting}
x = "hello there"
x.upper() # 'HELLO THERE'
x.split() # ['hello', 'there']
x.center(20) # '    hello there     '
x.endswith("e") # True
x.startswith("x") # False

t = "{} in 2016"
t.format("python", 2016) # 'python in 2016'
\end{lstlisting}

\subsection{Lists}
\texttt{[1,2,3]} or \texttt{list(1,2,3)}

\textit{Operations:} \texttt{+, *, len()}
\begin{lstlisting}
l = [3,2]
l.append(5) # l is [3, 2, 5]
l.extend(["a", "b"]) 
# l is now [3, 2, 5, 'a', 'b']
l.sort() # l is [2, 3, 5, 'a', 'b']
l.reverse() # l is ['b', 'a', 5, 3, 2]
[1,2] + [3,4] # [1, 2, 3, 4]
[1,2] * 2 # [1, 2, 1, 2]
5 in l # True
100 in l # False
\end{lstlisting}

\subsection{Dictionary}
\texttt{\{"name":"Turing", "age":30\}} or \\
\texttt{dict(name="Turing", age=30)} \\

\textit{Operations} \\
\texttt{[], del}

\begin{lstlisting}
d = dict(x=1, y=2)
list(d) # ['y', 'x']
list(d.items()) # [('y', 2), ('x', 1)]
list(d.values()) # [2, 1]
'x' in d # True
'a' in d # False
d['x'] # 1
d['t'] # raises KeyError
d.get('t', "Foo") # 'Foo'
d.get('y', "Foo") # 2
e = dict(a = 5, x = 10)
d.update(e) 
# d is now {'a': 5, 'y': 2, 'x': 10}
\end{lstlisting}

\subsection{Sets}
\texttt{set([1,2,3])}  or \texttt{\{1,2,3\}}

\textit{Operations}
\texttt{-, \&, |, \textasciicircum, len}

\begin{lstlisting}
u = {1,2}
v = {2,3}
u.union(v) # {1, 2, 3}
u | v # {1, 2, 3}
u & v # {2}
u.intersection(v) # {2}
u - v # {1}
u.isdisjoint(v) # False

w = {2}
u.issuperset(w) # True

y = {"a"}
u.isdisjoint(y) # True
\end{lstlisting}

\subsection{Tuples}

Similar to lists but can't be modified.

\begin{lstlisting}
l = (1,2,3)
l[2] = '0'
# Traceback (most recent call last):
#   File "<stdin>", line 1, in <module>
# TypeError: 'tuple' object does not 
# support item assignment
\end{lstlisting}


\noindent\rule{\linewidth}{0.05ex}
\section{Control structures}
\subsection{def}
Used to create functions

\begin{lstlisting}
def add(x, y):
  return x + y

t = add(5, 6)
print (t) # 11
\end{lstlisting}

\subsection{for}
Used for definite loops

\begin{lstlisting}
l = []
for i in [1,2,3]:
   l.append(i*2)
# l is [2,4,6]

m = []
for i in "ace":
   m.append(i)
# m is ['a', 'c', 'e']

s = 0
for i in range(1,20):
  s += i
# s is 190

\end{lstlisting}

\subsection{while}
Used for indefinite loops

\begin{lstlisting}
l = []
c = 10
while c > 0:
  l.append(c)
  c -= 2
# l will be [10, 8, 6, 4, 2]
\end{lstlisting}

\subsection{if}
Used for conditionals

\begin{lstlisting}
x = 10
if x%2 == 0:
   print ("{} is even".format(x))
# Will print "10 is even"

gwords = ["Axe.", "Axe", "axe"]:
for t in words:
  if t[0].isupper() and t.endswith("."):
     print ("Good")
  elif: t[0].isupper() or t.endswith(".")
     print ("Okay")  
  else:
     print ("Bad")
# Will print Good Okay Bad
\end{lstlisting}



\subsection{exceptions}
Used for error handling

\begin{lstlisting}
l = ["a", "b", "c"]
print (l[5])  # Raises an IndexError exception
try:
   print (l[5])
except IndexError as e:
   # We can handle the exception here. 
   print ("Out of bounds")
\end{lstlisting}
Mutliple exceptions with a common handler\\
\begin{lstlisting}
  except (IndexError, KeyError) as e:
\end{lstlisting}
Separate handlers for each type
\begin{lstlisting}
  try
    ...
  except Exception1 as e1:
    ...
  except Exception2 as e2:
    ...
  except Exception:
    # All other exceptions
\end{lstlisting}


\subsection{class}
Used to define classes

\begin{lstlisting}
class Person:
  def __init__(self, name, age):
    # Initialiser method
    self.name = name
    self.age = age

  def update_age(self, new_age):
    self.age = new_age

  def __str__(self):
    # Magic method to used for str()
    return "{} of age {}".format(self.name,
                                 self.age)


n = Person("Shams", 30)
print (str(n)) # Will print Shams of age 30
n.update_age(32)
print (str(n)) # Will print Shams of age 32
\end{lstlisting}


\section{Common Python idioms}
\begin{itemize}
\item Destructuring assignments
\begin{lstlisting}
a,b,c = 1,2,3
# a is 1, b is 2 and c is 3

l = [(1,2), (3,4), (5,6)]
for i,j in l:
   print (i+j)
# Will print 3, 7, 11
\end{lstlisting}
\item Iterating with count
  \begin{lstlisting}
l = "abcdef"
for i in range(len(l)):  # Bad style
   print (i, l[i])

for i,j in enumerate(l): # Good style
   print (i, j)
  \end{lstlisting}
\item List comprehensions
  \begin{lstlisting}
l = range(1, 5)
# l is [1,2,3,4]
m = [x**2 for x in l] 
# m is [1,4,9,16]
m = [x for x in l if x%2 == 0] 
# m is [1,3] (odd numbers)
m = [x**2 for x in l if x%2]
# m is [4, 16] (squares of even nos)
  \end{lstlisting}
\item Dictionary comprehensions
Similar to the above but creates dictionaries. Keys and values
separated by a \texttt{:}
  \begin{lstlisting}
d = dict (x = 2, y = 3)
{y:x for x,y in d.items()} # {2:'x', 3:'y'}
{x:y for x,y in d.items() if y%2} # {'x':2}
{x:0 for x in d} # {'x':0, 'y':0}
  \end{lstlisting}
\item Module importing
\begin{lstlisting}
if __name__ == "__main__":
  # Will only run when module is run
  f1()
else:
  # Will only run when module is imported
  f2()  
\end{lstlisting}

This can be used to get different behaviours when importing or running
a module
\end{itemize}


\section{Virtualenv}
\texttt{virtualenv} is a tool used to create isolated python
environments and then install $3^{rd}$ party packages inside them.

\begin{itemize}
\item Create a virtualenv using \texttt{python3 -m virtualenv
    /some/path}.
\item Activate the virtualenv using \\ \texttt{. /some/path/bin/activate}
  (Don't forget the initial \texttt{.}).
\item Now you can install packages inside the virtualenv using
  \texttt{pip} (e.g. \texttt{pip install requests}).
\item Get out of the virtualenv using \texttt{deactivate}.
\end{itemize}


\section{Modules}
A short list of commonly used modules. \\
\begin{tabular}[t]{| r | l |}
\hline
\cellcolor[gray]{0.9} Module & \cellcolor[gray]{0.9} Purpose \\
  re & Regular expressions \\
  os & OS calls (rm, stat etc.)\\
  sys & System information  \\
  json & Processing json documents \\
  urllib & HTTP client library \\
  logging & Displaying logs \\
  unittest & Testing your programs \\
  argparse & Command line parsing \\
  datetime & Date and time calculations \\
  subprocess & Executing shell commands \\
\hline
\end{tabular}


These are popular third party libraries. You can install them using \texttt{pip} inside a \texttt{virtualenv}.\\
\begin{tabular}[t]{| r | l |}
\hline
\cellcolor[gray]{0.9} Module & \cellcolor[gray]{0.9} Purpose \\
  psutil & Get system information \\
  pytest & Powerful test runner and API \\
  requests & HTTP Client with nice API \\
  coverage & Used to measure test coverage \\
  reportlab & PDF creation \\
  sqlalchemy & Database abstraction \\
  BeautifulSoup & HTML parsing \\
\hline
\end{tabular}






\begin{center}
\includegraphics[scale=0.4]{images/parthenon-callig.png}
\end{center}
For more information on courses which the Lyc\ae{}um conducts, please
visit \texttt{http://thelycaeum.in}. \\To stay notified on events and
courses, sign up for our mailer at \texttt{http://thelycaeum.in/resources.html}. \\For further
information, please email \texttt{noufal@nibrahim.net.in} or call
\textit{+91-9567091835}.
\vspace{0.5cm}

The Lyc\ae{}um conducts a 3 month long mentoring course for fresher
programmers and offers placement assistance. Please contact if you're
interested. The next batch starts in \rule{3cm}{0.1ex}.

\noindent\rule{\linewidth}{0.05ex}
\footnotesize All content \textcopyright The Lyc\ae{}um 2016. Version \input{version.tex}


\end{document}

